This page is still under construction.\hypertarget{_tuto_page_ref_hard}{}\section{Starting the hard way}\label{_tuto_page_ref_hard}
I know some people like to code the hard way. If you want to build everything by yourself, this is an example of how it is possible to build a mass function from scratch. It also presents how to combine mass functions. (Note that if you keep the -\/pedantic option in the makefile, you should write the comments using /$\ast$ comments instead of // comments.) 
\begin{DoxyCode}
 BF_BeliefFunction m, m2, m3;
 char* str = NULL;
 char bits[2], bits2[2];

 //A two bits element:
 bits[0] = 1;
 bits[1] = 1;
 
 //Another two bits element: 
 bits2[0] = 1;
 bits2[1] = 0;
 
 //Building a mass function the hard way: 
 m.elementSize = 2;
 m.nbFocals = 2;
 m.focals = malloc(sizeof(BF_FocalElement) * 2);
 m.focals[0].element = Sets_createElementFromBits(bits, 2);
 m.focals[0].beliefValue = 0.3;
 m.focals[1].element = Sets_createElementFromBits(bits2, 2);
 m.focals[1].beliefValue = 0.7;
 
 //Printing the mass function: 
 str = BF_beliefFunctionToBitString(m);
 printf("m = \n%s\n", str);
 free(str);
 
 //Combination of mass functions: 
 m2 = BF_DempsterCombination(m, m);
 m3 = BF_DempsterCombination(m2, m);
 
 //Printing the result of combination: 
 str = BF_beliefFunctionToBitString(m2);
 printf("m2 = \n%s\n", str);
 free(str);
 
 str = BF_beliefFunctionToBitString(m3);
 printf("m3 = \n%s\n", str);
 free(str);
 
 //Freeing mass functions: 
 BF_freeBeliefFunction(&m);
 BF_freeBeliefFunction(&m2);
 BF_freeBeliefFunction(&m3);
\end{DoxyCode}
\hypertarget{_tuto_page_Tuto_contact}{}\section{Contact}\label{_tuto_page_Tuto_contact}
Bastien Pietropaoli \par
 Ph.D. student \par
 INRIA, Rennes -\/ Bretagne Atlantique \par
 ACES Team \par


{\bfseries Email:} \par
 \href{mailto:bastien.pietropaoli@inria.fr}{\tt bastien.pietropaoli@inria.fr} \par
 \href{mailto:bastien.pietropaoli@gmail.com}{\tt bastien.pietropaoli@gmail.com} \par


Copyright 2011-\/2013, EDF. This software was developed with the collaboration of INRIA (Bastien Pietropaoli) 