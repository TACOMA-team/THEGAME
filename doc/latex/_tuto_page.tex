This page is still under construction.\hypertarget{_tuto_page_ref_hard}{}\section{Starting the hard way}\label{_tuto_page_ref_hard}
I know some people like to code the hard way. If you want to build everything by yourself, this is an example of how it is possible to build a mass function from scratch. It also presents how to combine mass functions. (Note that if you keep the -\/pedantic option in the makefile, you should write the comments using /($\ast$) comments instead of // comments.) 
\begin{DoxyCode}
\hyperlink{struct_b_f___belief_function}{BF\_BeliefFunction} m, m2, m3;
\textcolor{keywordtype}{char}* str = NULL;
\textcolor{keywordtype}{char} bits[2], bits2[2];

\textcolor{comment}{//A two bits element:}
bits[0] = 1;
bits[1] = 1;

\textcolor{comment}{//Another two bits element: }
bits2[0] = 1;
bits2[1] = 0;

\textcolor{comment}{//Building a mass function the hard way: }
m.elementSize = 2;
m.nbFocals = 2;
m.focals = malloc(\textcolor{keyword}{sizeof}(\hyperlink{struct_b_f___focal_element}{BF\_FocalElement}) * 2);
m.focals[0].element = \hyperlink{_sets_8c_a734e81728f1e2e9fab1b86045dcc42d4}{Sets\_createElementFromBits}(bits, 2);
m.focals[0].beliefValue = 0.3;
m.focals[1].element = \hyperlink{_sets_8c_a734e81728f1e2e9fab1b86045dcc42d4}{Sets\_createElementFromBits}(bits2, 2);
m.focals[1].beliefValue = 0.7;

\textcolor{comment}{//Printing the mass function: }
str = \hyperlink{_belief_functions_8c_a09113849ea5c4042faebfd9c2b53e1b5}{BF\_beliefFunctionToBitString}(m);
printf(\textcolor{stringliteral}{"m = \(\backslash\)n%s\(\backslash\)n"}, str);
free(str);

\textcolor{comment}{//Combination of mass functions: }
m2 = \hyperlink{_belief_functions_8c_a34a15102ee96fc30396ef8972b99c440}{BF\_DempsterCombination}(m, m);
m3 = \hyperlink{_belief_functions_8c_a34a15102ee96fc30396ef8972b99c440}{BF\_DempsterCombination}(m2, m);

\textcolor{comment}{//Printing the result of combination: }
str = \hyperlink{_belief_functions_8c_a09113849ea5c4042faebfd9c2b53e1b5}{BF\_beliefFunctionToBitString}(m2);
printf(\textcolor{stringliteral}{"m2 = \(\backslash\)n%s\(\backslash\)n"}, str);
free(str);

str = \hyperlink{_belief_functions_8c_a09113849ea5c4042faebfd9c2b53e1b5}{BF\_beliefFunctionToBitString}(m3);
printf(\textcolor{stringliteral}{"m3 = \(\backslash\)n%s\(\backslash\)n"}, str);
free(str);

\textcolor{comment}{//Freeing mass functions: }
\hyperlink{_belief_functions_8c_aa3c5107945acc34941aed1c7f7e04968}{BF\_freeBeliefFunction}(&m);
\hyperlink{_belief_functions_8c_aa3c5107945acc34941aed1c7f7e04968}{BF\_freeBeliefFunction}(&m2);
\hyperlink{_belief_functions_8c_aa3c5107945acc34941aed1c7f7e04968}{BF\_freeBeliefFunction}(&m3);
\end{DoxyCode}
\hypertarget{_tuto_page_Tuto_contact}{}\section{Contact}\label{_tuto_page_Tuto_contact}
Bastien Pietropaoli \par
 Ph.\-D. student \par
 I\-N\-R\-I\-A, Rennes -\/ Bretagne Atlantique \par
 A\-C\-E\-S Team \par


{\bfseries Email\-:} \par
 \href{mailto:bastien.pietropaoli@inria.fr}{\tt bastien.\-pietropaoli@inria.\-fr} \par
 \href{mailto:bastien.pietropaoli@gmail.com}{\tt bastien.\-pietropaoli@gmail.\-com} \par


Copyright 2011-\/2013, E\-D\-F. This software was developed with the collaboration of I\-N\-R\-I\-A (Bastien Pietropaoli) 